\documentclass[letterpaper,10pt]{article}

\usepackage{latexsym}
\usepackage[empty]{fullpage}
\usepackage{titlesec}
\usepackage{marvosym}
\usepackage[usenames,dvipsnames]{color}
\usepackage{verbatim}
\usepackage{enumitem}
\usepackage[hidelinks]{hyperref}
\usepackage{fancyhdr}
\usepackage[english]{babel}
\usepackage{tabularx}
\usepackage{fontawesome} 
\input{glyphtounicode}

\pagestyle{fancy}
\fancyhf{} % clear all header and footer fields
\renewcommand{\headrulewidth}{0pt}
\renewcommand{\footrulewidth}{0pt}

% Adjust margins
\addtolength{\oddsidemargin}{-0.5in}
\addtolength{\evensidemargin}{-0.5in}
\addtolength{\textwidth}{1in}
\addtolength{\topmargin}{-.5in}
\addtolength{\textheight}{1.0in}

\urlstyle{same}

\raggedbottom
\raggedright
\setlength{\tabcolsep}{0in}

% Sections formatting
\titleformat{\section}{
  \vspace{-4pt}\scshape\raggedright\huge
}{}{0em}{}[\color{black}\titlerule \vspace{-5pt}]

% Ensure that generate pdf is machine readable/ATS parsable
\pdfgentounicode=1

%-------------------------
% Custom commands
\newcommand{\resumeItem}[2]{
  \item\small{
    \textbf{#1}{: #2 \vspace{-2pt}}
  }
}

% Just in case someone needs a heading that does not need to be in a list
\newcommand{\resumeHeading}[4]{
    \begin{tabular*}{0.99\textwidth}[t]{l@{\extracolsep{\fill}}r}
      \textbf{#1} & #2 \\
      \textit{\small#3} & \textit{\small #4} \\
    \end{tabular*}\vspace{-5pt}
}

\newcommand{\resumeSubheading}[4]{
  \vspace{-1pt}\item
    \begin{tabular*}{0.97\textwidth}[t]{l@{\extracolsep{\fill}}r}
      \textbf{#1} & #2 \\
      \textit{\small#3} & \textit{\small #4} \\
    \end{tabular*}\vspace{-5pt}
}

\pdfgentounicode=1

\newcommand{\resumeSubSubheading}[2]{
    \begin{tabular*}{0.97\textwidth}{l@{\extracolsep{\fill}}r}
      \textit{\small#1} & \textit{\small #2} \\
    \end{tabular*}\vspace{-5pt}
}

\newcommand{\resumeSubItem}[2]{\resumeItem{#1}{#2}\vspace{-4pt}}

\renewcommand{\labelitemii}{$\circ$}

\newcommand{\resumeSubHeadingListStart}{\begin{itemize}[leftmargin=*]}
\newcommand{\resumeSubHeadingListEnd}{\end{itemize}}
\newcommand{\resumeItemListStart}{\begin{itemize}}
\newcommand{\resumeItemListEnd}{\end{itemize}\vspace{-5pt}}

%-------------------------------------------
%%%%%%  CV STARTS HERE  %%%%%%%%%%%%%%%%%%%%%%%%%%%%


\begin{document}

%----------HEADING-----------------
\begin{tabular*}{\textwidth}{l@{\extracolsep{\fill}}r}
  \textbf{\href{https://www.linkedin.com/in/abdul-rehman-8963bb236
/}{\huge Abdul Rehman}} 
  & \href{https://github.com/AnatoNamikaza}{\faGithub \hspace{1mm} github.com/AnatoNamikaza}\\
  {} & \href{https://www.linkedin.com/in/abdul-rehman-8963bb236/}{\faLinkedin \hspace{1mm} in/abdul-rehman-8963bb236}\\
  {} & \href{mailto:rehman.abdul.08@outlook.com}{\faEnvelope \hspace{1mm} rehman.abdul.08@outlook.com}\\
  {} & \faPhone \hspace{1mm} +923304051873 \\ 
\end{tabular*}

%-----------EDUCATION-----------------
\section{Education}
  \resumeSubHeadingListStart
    \resumeSubheading
      {Lahore University of Management Sciences (LUMS)}{Lahore, Pakistan}
      {Auditing, Computational Neuroscience}{Starting August 2024}
      \resumeItemListStart
        \resumeItem{Courses}
        {Computational Biology II, Applied Probability, Design and Algorithms, Distributed Systems, Deep Learning, Human Computer Interaction, Algorithmic Foundations of Big Tech}
        \resumeItem{Skills}
        {Computer Science, Health Informatics, Neuroscience, Bioinformatics, Proteomics, Machine Learning, Algorithms}
      \resumeItemListEnd

    \resumeSubheading
      {Forman Christian College (FCCU)}{Lahore, Pakistan}
      {Master's Degree, Computer Science}{Starting September 2024}
      \resumeItemListStart
        \resumeItem{Courses}
        {Data Mining, Advanced Machine Learning, Algorithmic Thinking, Research Methodology, Theory of Computing, Theory of Programming}
        \resumeItem{Skills}
        {Bioinformatics, Advanced Machine Learning}
      \resumeItemListEnd

    \resumeSubheading
      {National University of Computer and Emerging Sciences (FAST-NUCES)}{Lahore, Pakistan}
      {Bachelor's Degree, Computer Science}{August 2019 -- July 2023}
      \resumeItemListStart
        \resumeItem{Skills}
        {Computer Science}
      \resumeItemListEnd

    \resumeSubheading
      {Punjab Group of Colleges (PGC)}{Lahore, Pakistan}
      {Intermediate, Pre-Engineering}{March 2017 -- March 2019}
      \resumeItemListStart
        \resumeItem{Grade}
        {A+ (Merit-Scholarship Student)}
      \resumeItemListEnd

    \resumeSubheading
      {KIPS School \& Colleges}{Lahore, Pakistan}
      {Matric}{March 2015 -- March 2017}
      \resumeItemListStart
        \resumeItem{Grade}
        {A+ (Merit-Scholarship Student)}
      \resumeItemListEnd

  \resumeSubHeadingListEnd

%-----------EXPERIENCE-----------------
\section{Experience}

\resumeSubHeadingListStart

\resumeSubheading
{Lahore University of Management Sciences (LUMS)}{Lahore, Pakistan}
{Teaching Assistant (Part-time)}{Aug 2024 -- Present (10 months)}
\resumeItemListStart
  \resumeItem{Department}
  {Department of Computer Sciences, Department of Life Sciences}
  \resumeItem{Course}
  {Computational Biology II, Big Data Services and MLops}
  \resumeItem{Skills}
  {Python (Programming Language), Microsoft Azure, Databricks, Fabrics, Power BI}
\resumeItemListEnd

\resumeSubheading
{Lahore University of Management Sciences (LUMS)}{Lahore, Pakistan}
{Research Assistant (Full-time)}{Jul 2024 -- Present (11 months)}
\resumeItemListStart
  \resumeItem{Department}
  {Neuroscience, Biomedical Informatics \& Engineering Research Laboratory (BIRL), Department of Life Sciences}
  \resumeItem{Skills}
  {Health Informatics, Computer Science, Neuroscience}
\resumeItemListEnd

\resumeSubheading
{Forman Christian College (FCCU)}{Lahore, Pakistan}
{Research Assistant (Part-time)}{Sep 2024 -- Present (9 months)}
\resumeItemListStart
  \resumeItem{Department}
  {Department of Computer Sciences}
  \resumeItem{Skills}
  {Health Informatics, Neuroscience}
\resumeItemListEnd

\resumeSubheading
{National University of Computer and Emerging Sciences (FAST-NUCES)}{Lahore, Pakistan}
{Research Assistant (Part-time)}{Feb 2024 -- Feb 2025 (1 Year)}
\resumeItemListStart
  \resumeItem{Department}
  {Department of Computer Sciences}
  \resumeItem{Skills}
  {MATLAB, Python (Programming Language), Computer Science, IBM SPSS, Blender, OpenCV}
\resumeItemListEnd

\resumeSubheading
{National University of Computer and Emerging Sciences (FAST-NUCES)}{Lahore, Pakistan}
{Lab Instructor (On-site)}{Aug 2023 -- Jul 2024 (1 year)}
\resumeItemListStart
  \resumeItem{Department}
  {Department of Computer Sciences}
  \resumeItem{Taught Subjects}
  {Artificial Intelligence (2 sections), Data Mining (2 sections), Data Structures, Data Visualization and Analysis, Operating Systems (2 sections), Object-Oriented Programming, Programming Fundamentals}
  \resumeItem{Skills}
  {Computer Science, , Python (Programming Language), R (Programming Language), Linux, Weka, C++ / C (Programming Language), Lisp, Shell Scripting}
\resumeItemListEnd

\resumeSubheading
{National University of Computer and Emerging Sciences (FAST-NUCES)}{Lahore, Pakistan}
{Teaching Assistant (On-site)}{Aug 2023 -- Jul 2024 (1 year)}
\resumeItemListStart
  \resumeItem{Taught Subjects}
  {Blockchain, Data Science, Fundamentals of Computer Vision, Information Security, Internet of Things (2 sections), Natural Language Processing (2 sections), Quantum Computing (2 sections)}
  \resumeItem{Skills}
  {Python (Programming Language), C++, Computer Science, Qiskit, Arduino, Go (Programming Language), Proteus, Microsoft Azure, Microsoft SQL Server}
\resumeItemListEnd

\resumeSubheading
{Research Assistant | FAST-NUCES}{Lahore, Pakistan}
{Research Assistant}{Aug 2023 -- Jan 2024}
\resumeItemListStart
  \resumeItem{Research Areas}
  {Quantum Computing, Operations Research, Internet of Things, Cloud Computing, Computer Vision}
\resumeItemListEnd

\resumeSubheading
{Research Assistant | FAST-NUCES}{Lahore, Pakistan}
{Research Assistant}{Feb 2024 -- Jul 2024}
\resumeItemListStart
  \resumeItem{Research Areas}
  {Natural Language Processing, Internet Of Things, Research Operations}
\resumeItemListEnd

\resumeSubHeadingListEnd

%-----------RESEARCH AND DEVELOPMENT PROJECTS-----------------
\section{Research and Development Projects}
\resumeSubHeadingListStart

\resumeSubItem{Biomedical Informatics and Engineering Research Laboratory (Dr. Safee Ullah Chaudhary, PhD, LUMS)}
{Worked as a Research Assistant contributing to cross-disciplinary neuroscience and biomedical informatics projects integrating computer science, hardware systems, and AI techniques.
\begin{itemize}
    \item \textbf{BRACE Grant Proposal}: Contributed to grant writing, leading technical methodology sections, organizing pilot study documentation, and maintaining collaboration workflows using Mendeley and Dropbox.
    \item \textbf{SZ Brain Connectivity Analysis}: Conducted research on functional connectivity patterns in Schizophrenia (SZ) patients using fMRI datasets, focusing on network disruptions and brain region correlation analysis.
    \begin{itemize}
        \item \textbf{Dataset Handling}: Preprocessed fMRI datasets of SZ patients using SPM12, ensuring alignment, normalization, and artifact correction to maintain high data fidelity.
        \item \textbf{Connectivity Analysis}: Applied correlation matrices, ROI-based approaches, and seed-based methods to examine disrupted brain networks, particularly in the Default Mode Network (DMN) and Fronto-Parietal Network (FPN).
        \item \textbf{Technical Implementation}: Developed MATLAB scripts for brain connectivity visualization; explored emerging techniques like graph theory measures and machine learning classifiers to detect connectivity differences between SZ patients and healthy controls.
        \item \textbf{Reporting and Documentation}: Compiled research findings into organized reports, prepared figures for presentations, and drafted initial documentation for potential publication.
    \end{itemize}
    \item \textbf{ADHD Identification Project (Seiko Epson Corporation, Pakistan)}: Worked on a multi-phase ADHD identification project to detect ADHD in marginalized children and implement personalized interventions.
    \begin{itemize}
        \item \textbf{Phase 1}: Conducted real-time classroom monitoring using wearable sensors to assess students' Stress, Attention, and Distraction (SAD) scores.
        \item \textbf{Phase 2}: Used EEG for cross-validation of data obtained from wearable sensors, focusing on correlation between brain signals and classroom behaviors.
        \item \textbf{Phase 3}: Incorporated traditional psychological methods, including surveys and interviews, for final confirmation of ADHD diagnosis.
        \item \textbf{Goal}: Aimed to enhance ADHD detection accuracy and improve learning outcomes by providing tailored intervention strategies for children.
    \end{itemize}
\end{itemize}
}

\resumeSubItem{Emergency Expert (Dr. Ali Afzal, PhD, FAST-NU)}
{Contributed to the development of a real-time, AI-powered healthcare platform aimed at improving emergency medical access, diagnostics, and resource coordination across hospitals, pharmacies, and pharmaceutical companies in Pakistan, by providing them a android/iOS as well as web application, taking into consideration multiple factors such as literacy rate, negligence rate, efficiency of current medical services etc.
\begin{itemize}
    \item \textbf{AI Symptom Diagnostic System}: Designed and implemented a neural network-based diagnostic module that enabled patients to input symptoms through an intuitive binary-question interface. The system analyzed input patterns to provide rapid, high-confidence preliminary diagnoses and recommend appropriate triage levels, significantly reducing response time during emergency situations.
    \item \textbf{Multi-Platform Availability}: Designed and implemented both \textbf{web and mobile versions} of the platform to maximize accessibility and usability for diverse stakeholders, including patients, emergency responders, pharmacies, and healthcare administrators.
    \item \textbf{Emergency Routing and Hospital Integration}: Integrated Google Maps API to guide patients and ambulances to the nearest hospitals based on real-time factors like traffic conditions, ICU bed availability, oxygen supply, and on-call doctors.
    \item \textbf{Pharmacy Management Module}:
    \begin{itemize}
        \item Enabled pharmacies to manage inventory, track stock levels and expirations, and receive predictive analytics for supply-demand trends.
        \item Offered real-time dashboards for pharmacists to optimize ordering cycles and minimize medicine shortages.
    \end{itemize}
    \item \textbf{Pharmaceutical Company Panel}:
    \begin{itemize}
        \item Provided pharmaceutical companies with forecasting tools using anonymized demand data to improve production planning and supply chain logistics.
        \item Enabled region-wise tracking of medicine consumption trends and outbreak indicators.
    \end{itemize}
    \item \textbf{Stakeholder-Centric Architecture}: Built modular interfaces tailored for patients, hospitals, pharmacies, and pharma companies, ensuring role-specific access, data security, and smooth coordination across sectors.
    \item \textbf{Recognition and Achievement}: Shortlisted in the Ignite National Championship, standing out among over 25,000 Final Year Projects across Pakistan for innovation and societal impact in emergency healthcare.
    \item \textbf{Scalability and Future Vision}: Proposed future integration with ambulance dispatch systems, government health databases, and telemedicine platforms to expand functionality and national coverage.
\end{itemize}
}

\resumeSubItem{REX: The Self-Navigating AI-based Quadruped Robot (Dr. Arshad Ali, PhD, FAST-NU)}
{REX is a lightweight, cost-effective quadruped robot developed as an open-source alternative to Boston Dynamics’ Spot robot. Designed for adaptability and affordability, REX targets applications in search and rescue, exploration, surveillance, and future defense scenarios involving modular weaponization and reconnaissance.
\begin{itemize}
    \item \textbf{Project Objective}: To engineer a scalable, low-cost robotic platform capable of autonomous terrain navigation and real-time environment analysis, tailored for high-impact tasks in resource-constrained settings.
    \item \textbf{Simplified Hardware Architecture}: Built using \textbf{Arduino}, \textbf{servo motors}, and \textbf{ultrasonic sonar sensors} to keep the physical design minimal and cost-efficient. The compact hardware footprint contributed to improved agility and reduced manufacturing cost.
    \item \textbf{Cloud-Based Computation Model}:
    \begin{itemize}
        \item Offloaded heavy AI computations, image processing, and path planning to cloud servers using WiFi-enabled microcontrollers.
        \item Arduino handled only essential I/O tasks, transmitting sensory data to the cloud and executing lightweight instructions received in return.
        \item This architecture significantly reduced onboard computing requirements, lowering energy consumption and allowing for a more lightweight and modular body design.
    \end{itemize}
    \item \textbf{Autonomous Navigation and Terrain Adaptability}:
    \begin{itemize}
        \item Employed sonar and IMU sensors for obstacle detection and basic proprioception.
        \item Developed gait logic for terrain responsiveness, enabling the robot to adapt its step cycles based on incoming sensor data.
    \end{itemize}
    \item \textbf{Computer Vision and AI Integration}:
    \begin{itemize}
        \item Integrated \textbf{YOLO (You Only Look Once)} for object detection and obstacle classification, processed via cloud infrastructure.
        \item Implemented intelligent route optimization and object tagging for dynamic mission support.
    \end{itemize}
    \item \textbf{Software and Tools}:
    \begin{itemize}
        \item Used \textbf{Python}, \textbf{OpenCV}, and \textbf{TensorFlow Lite} for cloud-based inference.
        \item Simulated robot models using \textbf{Gazebo} and \textbf{ROS} to test movement behaviors and obstacle negotiation in virtual environments.
    \end{itemize}
    \item \textbf{Applications and Future Use Cases}: Designed for deployment in search and rescue missions, remote surveillance, and military operations. Its modular architecture supports future integration of robotic arms, cameras, and payloads.
\end{itemize}
}

\resumeSubItem{Hyper Learning Binary Political Optimizer (Dr. Maryam Bashir, PhD, FAST-NU)}
{HLBPO is a novel feature selection algorithm that enhances the original Political Optimizer by integrating hyper-learning strategies to improve search efficiency, avoid local minima, and deliver optimal feature subsets for classification tasks.
\begin{itemize}
    \item \textbf{Research Objective}: Addressed the challenge of dimensionality reduction in machine learning by developing an advanced feature selection technique that retains only the most informative features, boosting classification performance and minimizing computational overhead.
    \item \textbf{Algorithm Design}:
    \begin{itemize}
        \item Enhanced the Political Optimizer with a hyper-learning mechanism to guide exploration and exploitation phases more effectively, ensuring faster convergence and robust avoidance of local optima.
        \item Implemented adaptive update strategies based on population behavior, leading to improved stability and precision in feature subset selection.
    \end{itemize}
    \item \textbf{Experimental Evaluation}:
    \begin{itemize}
        \item Tested on 21 benchmark classification datasets from diverse domains.
        \item Outperformed nine state-of-the-art feature selection algorithms in terms of accuracy, feature selection ratio, and execution time.
        \item Achieved consistently lower feature selection ratios, validating HLBPO’s ability to isolate essential features while eliminating noise and redundancy.
    \end{itemize}
    \item \textbf{Tools and Technologies}:
    \begin{itemize}
        \item Algorithm development and experimentation conducted using \textbf{MATLAB}.
        \item \textbf{Power BI} was used for post-analysis visualization, performance benchmarking, and comparative result presentation.
        \item Evaluation metrics included classification accuracy, precision, recall, F1-score, and runtime efficiency across multiple classifiers.
    \end{itemize}
    \item \textbf{Impact and Applications}: Demonstrated HLBPO's practical viability for real-world machine learning pipelines, particularly in high-dimensional datasets common in biomedical diagnostics, finance, and text classification.
    \item \textbf{Research Outcome}: Compiled and documented findings for academic dissemination; proposed HLBPO as a lightweight yet powerful alternative to traditional metaheuristic-based feature selection methods.
\end{itemize}
}

\resumeSubItem{Multi-level, Multi-stage Lot-sizing and Scheduling in Flexible Flow Shop with Demand Information Updating (Dr. Hakeem Rehman, PhD, PU)}
{This research focuses on optimizing production planning in complex manufacturing environments—specifically the automotive industry—by addressing the NP-hard problem of lot-sizing and scheduling under dynamic demand conditions. A mixed-integer programming (MIP) model was developed and complemented by heuristic algorithms to ensure scalable, near-optimal solutions.
\begin{itemize}
    \item \textbf{Research Objective}: Minimize total production and inventory costs in a multi-product, multi-stage flexible flow shop system, accounting for uncertainty and updates in demand over time.
    \item \textbf{Mathematical Modeling}:
    \begin{itemize}
        \item Formulated a comprehensive \textbf{Mixed-Integer Programming (MIP)} model capturing interdependencies across multiple levels and production stages.
        \item Incorporated real-world manufacturing constraints such as setup times, capacity limitations, batch processing, and stage-wise precedence.
    \end{itemize}
    \item \textbf{Demand Forecasting and Uncertainty Modeling}:
    \begin{itemize}
        \item Utilized a \textbf{Martingale model of forecast evolution} to simulate realistic demand fluctuations and information updates over planning horizons.
        \item Integrated rolling horizon approaches to dynamically update scheduling decisions as new demand information becomes available.
    \end{itemize}
    \item \textbf{Heuristic Development and Performance Comparison}:
    \begin{itemize}
        \item Designed three tailored heuristics to tackle computational limitations of large-scale MIP solutions.
        \item \textbf{Heuristic 1} consistently outperformed the others in minimizing total costs and achieving high scheduling efficiency across multiple simulated environments.
    \end{itemize}
    \item \textbf{Tools and Implementation}:
    \begin{itemize}
        \item Modeled and solved the MIP using \textbf{Gurobi} and \textbf{AMPL}, while heuristic algorithms were implemented in \textbf{Python} and tested on synthetic datasets.
        \item Used \textbf{Excel} and \textbf{Power BI} for visualization, sensitivity analysis, and comparative performance reporting.
    \end{itemize}
    \item \textbf{Practical Implications}: Demonstrated the model’s effectiveness in reducing lead times, lowering inventory holding costs, and improving responsiveness in automotive production systems with demand variability.
    \item \textbf{Future Scope}: Proposed integration with real-time ERP data streams and IoT-based shop floor sensors for adaptive manufacturing intelligence.
\end{itemize}
}

\resumeSubItem{Qiskit Language Compiler (Dr. Faisal Aslam, PhD, FAST-NU)}
{This project focused on designing and implementing an enhanced quantum programming language compiler, intended to streamline quantum algorithm development and improve execution efficiency over the default Qiskit transpiler.
\begin{itemize}
    \item \textbf{Project Objective}: To optimize quantum circuit compilation by introducing a custom compiler layer on top of IBM’s \textbf{Qiskit SDK}, enhancing gate-level transformations, reducing circuit depth, and minimizing execution latency on real or simulated quantum hardware.
    \item \textbf{Compiler Enhancements}:
    \begin{itemize}
        \item Improved parsing of quantum code syntax to allow more intuitive user input and better compatibility with classical control structures.
        \item Introduced advanced optimization passes including gate cancellation, commutation-based reordering, and template matching for circuit simplification.
        \item Reduced the total number of gates and execution layers, leading to lower decoherence-related errors on noisy intermediate-scale quantum (NISQ) devices.
    \end{itemize}
    \item \textbf{Performance Evaluation}:
    \begin{itemize}
        \item Benchmarked against Qiskit’s native transpiler using test circuits such as quantum Fourier transforms (QFT), Grover’s algorithm, and variational quantum eigensolvers (VQE).
        \item Achieved measurable improvements in gate count reduction and overall compilation speed.
    \end{itemize}
    \item \textbf{Software and Tools}:
    \begin{itemize}
        \item Developed using \textbf{Python}, integrating with \textbf{Qiskit Terra}, \textbf{NumPy}, and \textbf{Matplotlib} for testing and visualization.
        \item Used \textbf{IBM Q Experience} simulators and cloud-based backends for validating execution efficiency.
    \end{itemize}
    \item \textbf{Impact and Utility}: Simplified the quantum software development pipeline for researchers and students by offering a more readable syntax and higher-performance circuit compilation, with potential for integration into educational toolchains or research platforms.
    \item \textbf{Future Scope}: Proposed extensions include support for hybrid quantum-classical workflows, adaptive compilation based on device topology, and integration with other frameworks such as Cirq and Braket.
\end{itemize}
}

\resumeSubItem{Telecom Identity Revealer}
{This large-scale data extraction and analysis project aimed to map ownership and usage patterns of telecom numbers across Pakistan. By leveraging publicly available and system-linked datasets, the tool successfully matched identities to mobile numbers, offering detailed insights into telecommunication behavior and data structuring across the region.
\begin{itemize}
    \item \textbf{Project Scope}: Processed over 240 million Pakistani phone numbers, successfully revealing ownership and registration details for approximately \textbf{180 million} entries with high accuracy.
    \item \textbf{Data Extracted}: For each identified number, the tool retrieved:
    \begin{itemize}
        \item \textbf{Full Name} of the registered owner,
        \item \textbf{National Identity (CNIC)} information,
        \item \textbf{Residential Address},
        \item \textbf{List of all active numbers} linked to the same identity.
    \end{itemize}
    \item \textbf{Technical Approach}:
    \begin{itemize}
        \item Employed custom scripts and query automation techniques for efficient data retrieval and validation.
        \item Applied normalization, deduplication, and mapping procedures to ensure dataset consistency and integrity.
        \item Developed a backend interface to facilitate secure querying and data presentation.
    \end{itemize}
    \item \textbf{Use Cases and Applications}:
    \begin{itemize}
        \item Useful for telecom fraud analysis, SIM misuse detection, digital identity verification, and regional usage analytics.
        \item Offers a macro-level view of mobile number distribution, identity overlaps, and carrier-specific behavior in Pakistan.
    \end{itemize}
    \item \textbf{Ethical Considerations}: Designed for controlled environments with strict data access protocols. Highlighted the need for stronger data privacy frameworks in national telecom infrastructure.
    \item \textbf{Impact}: Demonstrated the potential of structured telecom data mining to inform public safety initiatives, digital governance, and national-scale policy decisions.
\end{itemize}
}

\resumeSubHeadingListEnd


%--------PROGRAMMING SKILLS------------
\section{Programming Skills}
 \resumeSubHeadingListStart
    \resumeSubItem{Languages}
    {MATLAB, Python, R, C++/C/C\#, Lisp, Shell Scripting, Go, Qiskit, Arduino, JavaScript, Shell Script, SQL, Assembly, HTML/CSS, AMPL, Rust, Kotlin, Swift}
    \resumeSubItem{Tools}
    {Arduino, Blender, Visual Studio, Visual Studio Code, Azure, Postman, MSSQL, MySQL, Figma, Adobe Illustrator, Github Desktop, Matlab, MRIcro, RPNext, Weka, Linux, Proteus, Qiskit, IBM SPSS, EEGLAB, BCI, Anaconda, Databricks, Ganache, Kali Linux, YOLO, Wireshark, NASM/MASM, Mingw, Logic Works, TinkerCAD, Virtual Box, Android Studio, Xcode, \LaTeX, Google Colab, Git}
    \resumeSubItem{Frameworks}
    {Angular, MERN, Flutter, React, Django, Flask, Node.js, TensorFlow, PyTorch, Express.js, Spring Boot, .NET, Vue.js, Next.js}
 \resumeSubHeadingListEnd

\section{Languages}
 \resumeSubHeadingListStart
    \resumeSubItem{Urdu}{Native Proficiency}
    \resumeSubItem{Punjabi}{Native Proficiency}
    \resumeSubItem{English}{Professional Proficiency (IELTS: 8 Band)}
 \resumeSubHeadingListEnd

\section{Leadership}
 \resumeSubHeadingListStart
    \resumeSubheading
        {Research Assistant, Biomedical Informatics \& Engineering Research Lab}{2024}
        {Lahore University of Management Sciences (LUMS)}{Lahore, Pakistan}
    \resumeSubheading
        {Head Officer, Robo Rumble Dept, SOFTEC}{2022}
        {National University of Computer \& Emerging Sciences (FAST-NU)}{Lahore, Pakistan}
 \resumeSubHeadingListEnd

%-------------------------------------------
\end{document}
